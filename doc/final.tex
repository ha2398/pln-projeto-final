\documentclass[sigconf]{acmart}

\usepackage{booktabs}
\usepackage[brazil]{babel} 
\usepackage[utf8]{inputenc}

%Conference
\acmConference[DCC/UFMG]{Projeto Final - DCC030}{Dezembro 2017}{Belo 
Horizonte, MG} 
\acmYear{2017}
\copyrightyear{2017}
\acmArticle{4}
\acmPrice{15.00}

\begin{document}
\title{Moodic}
\subtitle{Projeto final da disciplina Processamento de Linguagem
Natural (DCC/UFMG)}

\author{Hugo Araujo de Sousa}
\affiliation{%
  \institution{Universidade Federal de Minas Gerais}
  \streetaddress{Instituto de Ciências Exatas}
  \city{Belo Horizonte} 
  \state{MG}
}
\email{hugosousa@dcc.ufmg.br}


\begin{abstract}
  
\end{abstract}

%
% The code below should be generated by the tool at
% http://dl.acm.org/ccs.cfm
% Please copy and paste the code instead of the example below. 
%
\begin{CCSXML}
<ccs2012>
<concept>
<concept_id>10010147.10010178.10010179.10003352</concept_id>
<concept_desc>Processamento de Linguagem Natural~Análise de Sentimentos</concept_desc>
<concept_significance>400</concept_significance>
</concept>

<concept>
<concept_id>10010147.10010178.10010179.10003352</concept_id>
<concept_desc>Metodologias computacionais~Extração de informação</concept_desc>
<concept_significance>100</concept_significance>
</concept>

<concept>
<concept_id>10010405.10010469.10010475</concept_id>
<concept_desc>Computação aplicada~Computação de som e música</concept_desc>
<concept_significance>100</concept_significance>
</concept>
</ccs2012>
\end{CCSXML}

\ccsdesc[400]{Processamento de Linguagem Natural~Análise de Sentimentos}
\ccsdesc[100]{Metodologias computacionais~Extração de informação}
\ccsdesc[100]{Computação aplicada~Computação de som e música}

\maketitle

\section{Introdução}

\section{Contexto}

\section{Modelagem}

\section{Implementação}

\section{Estudo de caso}

\section{Conclusão}

\bibliographystyle{ACM-Reference-Format}
\bibliography{bibliography} 

\end{document}
